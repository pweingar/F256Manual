\chapter{Versatile Interface Adapter}

The \jr\ includes a Western Design Center WDC65C22 versatile interface adapter or VIA. The VIA provides several useful features for I/O and timing:

\begin{itemize}
    \item Two independent I/O ports of eight parallel bits (PA, and PB).

    \item Four handshake control lines (CA1, CA2, CB1, and CB2)

    \item Programmable serial register for serial I/O operations

    \item Two independent timer counters
\end{itemize}

On the \jr, the VIA is connected to header which is compatible with the keyboard header on the Commodore VIC-20 and C-64. This means that a Commodore compatible keyboard could be connected to the \jr\ and used for keyboard input with appropriate programming. The VIA also provides access to the two Atari-style joystick ports. The pins could also be used for general purpose I/O, although the voltage levels are for 3.3 volt logic instead of the 5 volt logic used in older 8-bit machines.

A complete description of the VIA would be rather long, so this guide will merely list out the register addresses and provide a quick break-down on the register functions. For a complete description, please see the data sheet from Western Design Center. See table~\ref{tab:via_reg} for a listing of all the VIA registers.

\begin{table}[ht]
    \begin{center}
        \begin{tabular}{|c|c|c|l|} \hline
            Address & R/W & Name & Purpose \\\hline\hline
            \verb+0xDC00+ & R/W & IORB & Port B data \\\hline
            \verb+0xDC01+ & R/W & IORA & Port A data \\\hline
            \verb+0xDC02+ & R/W & DDRB & Port B Data Direction Register \\\hline
            \verb+0xDC03+ & R/W & DDRA & Port A Data Direction Register \\\hline
            \verb+0xDC04+ & R/W & T1C\_L & Timer 1 Counter Low \\\hline
            \verb+0xDC05+ & R/W & T1C\_H & Timer 1 Counter High \\\hline
            \verb+0xDC06+ & R/W & T1L\_L & Timer 1 Latch Low \\\hline
            \verb+0xDC07+ & R/W & T1L\_H & Timer 1 Latch High \\\hline
            \verb+0xDC08+ & R/W & T2C\_L & Timer 2 Counter Low \\\hline
            \verb+0xDC09+ & R/W & T2C\_H & Timer 2 Counter High \\\hline
            \verb+0xDC0A+ & R/W & SDR & Serial Data Register \\\hline
            \verb+0xDC0B+ & R/W & ACR & Auxiliary Control Register \\\hline
            \verb+0xDC0C+ & R/W & PCR & Peripheral Control Register \\\hline
            \verb+0xDC0D+ & R/W & IFR & Interrupt Flag Register \\\hline
            \verb+0xDC0E+ & R/W & IER & Interrupt Enable Register \\\hline
            \verb+0xDC0F+ & R/W & IORA2 & Port A data (no handshake) \\\hline
        \end{tabular}
    \end{center}
    \caption{VIA Registers}
    \label{tab:via_reg}
\end{table}

\begin{table}[ht]
    \begin{center}
        \begin{tabular}{|c|c|c|c|c|c|c|c|c|} \hline
            Name & 7 & 6 & 5 & 4 & 3 & 2 & 1 & 0 \\\hline\hline
            ACR & \multicolumn{2}{|c|}{T1\_CTRL} & T2\_CTRL & \multicolumn{3}{|c|}{SR\_CTRL} & PBL\_EN & PAL\_EN \\\hline
            PCR & \multicolumn{3}{|c|}{CB2\_CTRL} & CB1\_CTRL & \multicolumn{3}{|c|}{CA2\_CTRL} & CA1\_CTRL \\\hline
            IFR & IRQF & T1F & T2F & CB1F & CB2F & SRF & CA1F & CA2F \\\hline
            IER & SET & T1E & T2E & CB1E & CB2E & SRE & CA1E & CA2E \\\hline
        \end{tabular}
    \end{center}
    \caption{VIA Control Registers}
    \label{tab:via_ctrl_reg}
\end{table}

\begin{description}
    \item[IORA] Input/Output Register for Port A. The eight bits correspond to the eight pins on port A.

    \item[DDRA] Data Direction Register for Port A. Each bit configures the corresponding pin to be input (0) or output (1).

    \item[IORB] Input/Output Register for Port B. The eight bits correspond to the eight pins on port B.

    \item[DDRB] Data Direction Register for Port B. Each bit configures the corresponding pin to be input (0) or output (1).

    \item[T1C\_L, T1C\_H] Timer 1 counter value

    \item[T1L\_L, T1L\_H] Timer 1 latch

    \item[T2C\_L, T2C\_H] Timer 2 counter value

    \item[SDR] is the shift register. Serial input may be read here, or data may be written here to be shifted out.

    \item[ACR] Auxiliary Control Register. Contains fields to control the function of timer 1, timer 2, the shift register, and how Port A and Port B latch data. See table~\ref{tab:via_ctrl_reg} for details.

    \item[PCR] Peripheral Control Register. Contains fields to control how the CA1, CA2, CB1, and CB2 handshake pins are used. See table~\ref{tab:via_ctrl_reg} for details.

    \item[IFR] Interrupt Flag Register. Contains flags indicating which condition triggered an interrupt request. Possible conditions are timer 1, timer 2, CB1, CB2, CA1, CA2, and shift register complete. See table~\ref{tab:via_ctrl_reg} for details.

    \item[IER] Interrupt Enable Register. Contains flags to enable or disable interrupts based on the different possible conditions. See table~\ref{tab:via_ctrl_reg} for details.

    \item[IORA2] Same as IOPA except that the built-in handshaking capability is not used.

\end{description}

\section*{Joystick Support}

The \jr\ has two IDC headers that can be connected to a DB-9 socket to allow Atari style joysticks to be used (see figure:~\ref{fig:joystick_ports} for the pinouts). Joystick header 0 is wired to the pins of Port B, and joystick header 1 is connected to Port A. The various joystick switches are connected to the ports in same manner as on the C-64, with the exception that more buttons are supported (see table:~\ref{tab:via_joystick}).

\begin{figure}[ht]
    \begin{center}
        \includegraphics[scale=0.65]{images/f256_port_joystick.pdf}
    \end{center}
    \caption{Joystick Port Pinouts}
    \label{fig:joystick_ports}
\end{figure}

In order to use the joysticks, the DDR bits for the ports must be set to 0 for input. Then the input/output register for the port may be read. If a button or switch is closed on the joystick, the corresponding bit in the I/O register will be clear (0). If the button is not pressed, the bit will be set (1).

As a reminder: be aware that the WDC65C22 on the \jr\ is being used with a 3.3 volt supply. This means that any device plugged into the joystick ports should be 3.3 volt tolerant and should not raise any pin above 3.3 volts. Otherwise damage could occur.

\begin{table}[ht]
    \begin{center}
        \begin{tabular}{|c|c|c|c|c|c|c|c|} \hline
            7 & 6 & 5 & 4 & 3 & 2 & 1 & 0 \\\hline\hline
            --- & BUTTON2 & BUTTON1 & BUTTON0 & RIGHT & LEFT & DOWN & UP \\ \hline
        \end{tabular}
    \end{center}
    \caption{Joystick Flags}
    \label{tab:via_joystick}
\end{table}

\example{Displaying Joystick 1}
In this example, we will poll joystick 1 and print out the state of all the buttons by printing the byte we read from the joystick port as a simple binary number. The example will try to be a little bit smart by only printing the value when the value has changed. NOTE: this example expects OpenKernal to be installed, and will call two of its routines for initializing the screen and printing a character.

First, we initialize the screen, the variable we use to track the old value of the joystick port, and the VIA (setting port A to be an input port):
\begin{verbatim}
ok_cint = $FF81                         ; OpenKernal routine to initialize the screen
ok_cout = $FFD2                         ; OpenKernal routine to print a character in A

; Variables

* = $0080

value:      .byte ?                     ; Variable to store the previous value of the joystick
prv:        .byte ?                     ; Copy of value for printing

* = $e000

start:      jsr ok_cint                 ; Set up the screen

            lda #$FF                    ; Set the previous value to $FF
            sta value

            stz MMU_IO_CTRL             ; Switch to I/O Page 0

            lda #$00                    ; Set VIA Port A to input
            sta VIA_DDRA
\end{verbatim}

Next, we print the OpenKernal code to clear the screen, and we print out the byte in \verb+value+ as a binary number.

\begin{verbatim}
loop1:      lda #147                    ; Print the CBM clear screen code
            jsr ok_cout

            lda value                   ; Copy the value to prv
            sta prv

            ldx #8                      ; Loop for all eight bits
loop2:      asl prv                     ; Shift MSB into the carry
            bcc is0                     ; If it's 0, print '0'

            lda #'1'                    ; Otherwise, print '1'
            jsr ok_cout
            bra repeat                  ; And go to the next bit

is0:        lda #'0'                    ; Print '0'
            jsr ok_cout

repeat:     dex                         ; Count down
            bne loop2                   ; Repeat until we've done all 8 bits
\end{verbatim}

Next, we read the value of port A. If it is different from \verb+value+, we save it to \verb+value+ and go back to print the byte we read. Otherwise, we keep waiting and polling the joystick port.

\begin{verbatim}
            stz MMU_IO_CTRL             ; Switch to I/O Page 0

wait:       lda VIA_IORA                ; Get the status of port A
            cmp value                   ; Is it different from before?
            beq wait                    ; Yes: keep waiting

            sta value                   ; Save this value as the previous one
            bra loop1                   ; And go to print it
\end{verbatim}
